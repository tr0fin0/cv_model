\documentclass[class=article, crop=false]{standalone}

\begin{document}
\section*{Résumé}
\summary{Je m'appelle Guilherme, un étudiant brésilien de 23 ans de l'UNICAMP, poursuivant depuis 2022 un double diplôme de 3 ans à l'ENSTA Paris au programme de master.}{J'effectue actuellement mon stage de 2ème année de césure chez Sysnav développant et validant des systèmes embarqués avec technologie de navigation magnéto-inertielle.}


\section*{Expérience}
\topicblock{Embedded Hardware Designer Intern}{\href{https://www.sysnav.fr/}{Sysnav}}{Vernon, France}{04/2024 -- 09/2024}{Expérience dans le test, la validation et l'étude des systèmes électroniques. Compétent dans la conception et la conception de systèmes électroniques, offrant des contributions précieuses au succès du projet}

\topicblock{Retail Vendor Manager Intern}{\href{https://www.aboutamazon.fr/}{Amazon France}}{Clichy, France}{09-2023 -- 03/2024}{En collaboration avec un Senior Vendor Manager, j'ai aidé à fournir des outils conçus pour optimiser les performances commerciales et stimuler la croissance des ventes. Cela incluait un macro Excel conçu pour identifier la sélection sous licence vingt fois plus vite, permettant des analyses approfondies dans de grands ensembles de données, découvrant des tendances et extrayant des insights clients.\\
        
J'ai également développé un dashboard complet et visuellement attrayant dans Excel, facilitant le benchmarking à différents niveaux de granularité, simplifiant l'identification des opportunités de croissance et la mise en œuvre de nouvelles stratégies pour l'équipe.}

\topicblock{Neuroscience Research Intern}{\href{https://www.sorbonne-universite.fr/}{Sorbonne Université}}{Paris, France}{05/2023 -- 08/2023}{Développement d'un environnement de simulation utilisant Webots et Python, facilitant l'évaluation de la modélisation sensorimotrice de l'hippocampe basée sur le cadre proposé par Benoît Girard et Sylvain Argentieri}

\topicblock{Head of Team}{\href{}{Phoenix Robotics Team of UNICAMP}}{Campinas, Brésil}{07/2021 -- 07/2022}{Fondée en 2001 par des étudiants en génie mécatronique, l'équipe Phoenix Robotic comptait, en 2022, 83 étudiants de premier cycle issus de différentes filières concevant, construisant et concourant avec des robots autonomes et radiocommandés dans 12 catégories différentes.\\
        
Définition d'une planification à long terme pour les départements et projets de l'équipe visant une croissance durable, ce qui a conduit à une nouvelle culture axée sur les personnes et à la définition d'objectifs à court et moyen terme ainsi que de délais pour les 5 coordinateurs de zone et les 12 coordinateurs de projet.\\

Amélioration du processus de sélection des membres de l'équipe, visant une plus grande diversité, ce qui a abouti au meilleur ratio d'inscription, plus de 125 personnes sur 500, et au meilleur pourcentage de femmes parmi les approuvés, 16 sur 35 personnes.}


\section*{Éducation}
\topicblock{MSc Ingénierie | Double Diplôme}{\href{https://www.ensta-paris.fr/}{ENSTA Paris}}{Palaiseau, France}{07/2022 -- En Cours}{Informatique -- Robotique}

\topicblock{BSc Engineering | Double degree}{\href{https://www.unicamp.br/unicamp/universidade}{UNICAMP}}{Campinas, Brésil}{03/2019 -- En Cours}{Contrôle et Automatisation -- Mécatronique}


\vspace{7.5mm}
\begin{multicols}{2}
    \section*{Langues}
    \topicline{Portugais}{Langue Maternelle}
    \topicline{Français}{Avancé C1}
    \topicline{Anglais}{Avancé C1}
    \columnbreak
    \section*{Skills}
    \topicline{Programmation}{C, LaTeX, MATLAB, Python et VBA}
    \topicline{Office}{Excel, PowerPoint et Word}
    \topicline{Autres}{Altium Designer, Adobe Illustrator et Photoshop}
\end{multicols}
\end{document}