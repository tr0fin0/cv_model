\documentclass[10pt, a4paper]{extarticle}

\usepackage[margin=0.5in]{geometry} % Adjusts the margins

\usepackage{multicol}
\usepackage{mdwlist}
\usepackage{relsize}
\usepackage{hyperref}
\usepackage{xcolor}
\definecolor{dark-blue}{rgb}{0.15,0.15,0.4}
\hypersetup{colorlinks, linkcolor={dark-blue}, citecolor={dark-blue}, urlcolor={dark-blue}} % Assigns the dark blue color to all links in the template

\usepackage{tgpagella} % Use the TeX Gyre Pagella font throughout the document
\usepackage[T1]{fontenc}
\usepackage{microtype} % Slightly tweaks character and word spacings for better typography

\pagestyle{empty} % Stop page numbering

%----------------------------------------------------------------------------------------
%	DEFINE STRUCTURAL COMMANDS
%----------------------------------------------------------------------------------------

\newenvironment{indentsection} % Defines the indentsection environment which indents text in sections titles
{\begin{list}{}{\setlength{\leftmargin}{\newparindent}\setlength{\parsep}{0pt}\setlength{\parskip}{0pt}\setlength{\itemsep}{0pt}\setlength{\topsep}{0pt}}}{\end{list}}

\newcommand*\maintitle[2]{\noindent{\LARGE \textbf{#1}}\ \ \ \emph{#2}\vspace{0.3em}} % Main title (name) with date of birth or subtitle

\newcommand*\roottitle[1]{\subsection*{#1}\vspace{-0.3em}\nopagebreak[4]} % Top level sections in the template

\newcommand{\headedsection}[3]{\nopagebreak[4]\begin{indentsection}\item[]\textscale{1.1}{#1}\hfill#2#3\end{indentsection}\nopagebreak[4]} % Section title used for a new employer

\newcommand{\headedsubsection}[3]{\nopagebreak[4]\begin{indentsection}\item[]\textbf{#1}\hfill\emph{#2}#3\end{indentsection}\nopagebreak[4]} % Section title used for a new position

\newcommand{\bodytext}[1]{\nopagebreak[4]\begin{indentsection}\item[]#1\end{indentsection}\pagebreak[2]} % Body text (indented)

\newcommand{\inlineheadsection}[2]{\begin{basedescript}{\setlength{\leftmargin}{\doubleparindent}}\item[\hspace{\newparindent}\textbf{#1}]#2\end{basedescript}\vspace{-1.7em}} % Section title where body text starts immediately after the title

\newcommand*\acr[1]{\textscale{.85}{#1}} % Custom acronyms command

\newcommand*\bull{\ \ \raisebox{-0.365em}[-1em][-1em]{\textscale{4}{$\cdot$}} \ } % Custom bullet point for separating content

\newlength{\newparindent} % It seems not to work when simply using \parindent...
\addtolength{\newparindent}{\parindent}

\newlength{\doubleparindent} % A double \parindent...
\addtolength{\doubleparindent}{\parindent}

\newcommand{\breakvspace}[1]{\pagebreak[2]\vspace{#1}\pagebreak[2]} % A custom vspace command with custom before and after spacing lengths
\newcommand{\nobreakvspace}[1]{\nopagebreak[4]\vspace{#1}\nopagebreak[4]} % A custom vspace command with custom before and after spacing lengths that do not break the page

\newcommand{\spacedhrule}[2]{\breakvspace{#1}\hrule\nobreakvspace{#2}} % Defines a horizontal line with some vertical space before and after it
\newcommand{\areaofinterest}{Robotics, Electronics, or PCB design}



\begin{document} 


\maintitle{Guilherme Nunes Trofino}{[EN version]} 

\noindent\href{mailto:guitrofino@gmail.com}{guitrofino@gmail.com}
\qquad\href{https://www.linkedin.com/in/guilherme-trofino/}{LinkedIn: guilherme-trofino}
\qquad\href{https://github.com/tr0fin0}{GitHub: tr0fin0}
\qquad+33 07 49 18 55 00



\roottitle{Summary}
\begin{multicols}{2}
\noindent \textit{I'm Guilherme, a 23-year-old Brazilian student from UNICAMP, pursuing since 2022 a 3-year double degree exchange program at ENSTA Paris master's program.}

\columnbreak

\noindent I'm looking for a 5-month internship from the 25th of March until the 9th of September in \areaofinterest.
\end{multicols}


\roottitle{Experience}
\headedsection
    {Retail Vendor Manager Intern}
    {}
    {
        \headedsubsection
        {\href{https://www.aboutamazon.fr/}{Amazon France}}
        {Clichy, France \qquad 09/2023 -- 03/2024}
        {\bodytext{In collaboration with a Senior Vendor Manager, I helped provide tools designed to optimize business performance and drive sales growth. This included an Excel macro designed to identify licensed selection twenth times faster, allowing deep dives in large datasets, uncovering trends, extracting customer insights, and taking action.\\
        
        \noindent I also developed a comprehensive and visually appealing Dashboard in Excel, facilitating benchmarking at various levels of granularity, simplifying the identification of growth opportunities and the implementation of new strategies for the team.}}
    }

\headedsection
    {Neuroscience Research Intern}
    {}
    {
    \headedsubsection
        {\href{https://www.isir.upmc.fr/?lang=en}{ISIR}, \href{https://www.sorbonne-universite.fr/}{Sorbonne Université}}
        {Paris, France \qquad 05/2023 -- 08/2023}
        {\bodytext{Developing a user-friendly simulation environment using Webots and Python, facilitating the assessment of the sensorimotor modelization of the hippocampus based on the framework proposed by Benoît Girard and Sylvain Argentieri.}}
    }

\headedsection
    {Head of Team}
    {}
    {
        \headedsubsection
        {\href{}{Phoenix Robotics Team of UNICAMP}}
        {Campinas, Brazil \qquad 07/2021 -- 07/2022}
        {\bodytext{Established in 2001 by mechatronics engineering students, the Phoenix Robotic Team had, in 2022, 83 undergraduate students from different courses designing, building, and competing with autonomous and radio-controlled robots across 12 categories.\\
            
        \noindent Definition of long-term planning for the Team's departments and projects aiming for sustainable growth, resulting in a new people-driven culture and mission setting short and medium-term targets and deadlines for the 5 area coordinators and the 12 project coordinators.\\
        
        \noindent Improve Team Members' selection process, aiming for more diversity, resulting in the best inscription ratio, over 125 people from 500, and resulting in the best percentage of women among the approved, 16 from 35 people.}}
    }

\headedsection
    {Head of Electronics Department}
    {}
    {
        \headedsubsection
        {\href{}{Phoenix Robotics Team of UNICAMP}}
        {Campinas, Brazil \qquad 08/2020 -- 07/2021}
        {\bodytext{
        During this timeframe, we closed a strategic partnership with Altium, securing access to Altium Designer and Altium 365, and we implemented a 10-week project-based training program, onboarding 12 people to the team.
            }
        }
    }


\roottitle{Education}
\headedsection
    {MSc Engineering | Double degree}
    {}
    {
        \headedsubsection
        {\href{https://www.ensta-paris.fr/}{ENSTA Paris}, \href{https://www.ip-paris.fr/}{IP Paris}}
        {Palaiseau, France \qquad 07/2022 -- Present}
        {\bodytext{Computer Science -- Robotics}}
    }

\headedsection
    {BSc Engineering | Double degree}
    {}
    {
    \headedsubsection
        {\href{https://www.fem.unicamp.br/index.php/pt-br/}{FEM}, \href{https://www.unicamp.br/unicamp/universidade}{UNICAMP}}
        {Campinas, Brazil \qquad 03/2019 -- Present}
        {\bodytext{Control and Automation -- Mechatronics}}
    }


\vspace{6mm}
\begin{multicols}{2}
    \roottitle{Languages}
    \vspace{2mm}
    \textbf{Portuguese}: Native\\
    \textbf{French}: Advanced C1\\
    \textbf{English}: Advanced C1\\
    \columnbreak
    \roottitle{Skills}
    \vspace{2mm}
    \textbf{Programming}: C, LaTeX, MATLAB, Python and VBA\\
    \textbf{Office}: Excel, PowerPoint and Word\\
    \textbf{Others}: Altium Designer, Adobe Illustrator and Photoshop\\
\end{multicols}



\newpage
\maintitle{Guilherme Nunes Trofino}{[FR version]}

\noindent\href{mailto:guitrofino@gmail.com}{guitrofino@gmail.com}
\qquad\href{https://www.linkedin.com/in/guilherme-trofino/}{LinkedIn: guilherme-trofino}
\qquad\href{https://github.com/tr0fin0}{GitHub: tr0fin0}
\qquad\href{tel:+330749185500}{+33 07 49 18 55 00}


\roottitle{Résumé}
\begin{multicols}{2}
\noindent \textit{Je m'appelle Guilherme, un étudiant brésilien de 23 ans de l'UNICAMP, poursuivant depuis 2022 un double diplôme de 3 ans à l'ENSTA Paris au programme de master.}

\columnbreak

\noindent Je recherche une stage de césure de 5 mois entre le 25 mars jusqu'à le 9 septembre en \areaofinterest.
\end{multicols}


\roottitle{Expérience}
\headedsection
    {Retail Vendor Manager Intern}
    {}
    {
        \headedsubsection
        {\href{https://www.aboutamazon.fr/}{Amazon France}}
        {Clichy, France \qquad 09/2023 -- 03/2024}
        {\bodytext{En collaboration avec un Senior Vendor Manager, j'ai aidé à fournir des outils conçus pour optimiser les performances commerciales et stimuler la croissance des ventes. Cela incluait un macro Excel conçu pour identifier la sélection sous licence vingt fois plus vite, permettant des analyses approfondies dans de grands ensembles de données, découvrant des tendances et extrayant des insights clients.\\
        
        \noindent J'ai également développé un dashboard complet et visuellement attrayant dans Excel, facilitant le benchmarking à différents niveaux de granularité, simplifiant l'identification des opportunités de croissance et la mise en œuvre de nouvelles stratégies pour l'équipe.}}
    }

\headedsection
    {Neuroscience Research Intern}
    {}
    {
        \headedsubsection
        {\href{https://www.isir.upmc.fr/?lang=en}{ISIR}, \href{https://www.sorbonne-universite.fr/}{Sorbonne Université}}
        {Paris, France \qquad 05/2023 -- 08/2023}
        {\bodytext{Développement d'un environnement de simulation utilisant Webots et Python, facilitant l'évaluation de la modélisation sensorimotrice de l'hippocampe basée sur le cadre proposé par Benoît Girard et Sylvain Argentieri.}}
    }

\headedsection
    {Head of Team}
    {}
    {
        \headedsubsection
        {\href{}{Phoenix Robotics Team of UNICAMP}}
        {Campinas, Brazil \qquad 07/2021 -- 07/2022}
        {\bodytext{Fondée en 2001 par des étudiants en génie mécatronique, l'équipe Phoenix Robotic comptait, en 2022, 83 étudiants de premier cycle issus de différentes filières concevant, construisant et concourant avec des robots autonomes et radiocommandés dans 12 catégories différentes.\\
        
        \noindent Définition d'une planification à long terme pour les départements et projets de l'équipe visant une croissance durable, ce qui a conduit à une nouvelle culture axée sur les personnes et à la définition d'objectifs à court et moyen terme ainsi que de délais pour les 5 coordinateurs de zone et les 12 coordinateurs de projet.\\

        \noindent Amélioration du processus de sélection des membres de l'équipe, visant une plus grande diversité, ce qui a abouti au meilleur ratio d'inscription, plus de 125 personnes sur 500, et au meilleur pourcentage de femmes parmi les approuvés, 16 sur 35 personnes.}}
    }

\headedsection
    {Head of Electronics Department}
    {}
    {        
        \headedsubsection
        {\href{}{Phoenix Robotics Team of UNICAMP}}
        {Campinas, Brazil \qquad 08/2020 -- 07/2021}
        {\bodytext{Au cours de cette période, nous avons conclu un partenariat stratégique avec Altium, garantissant l'accès à Altium Designer et à Altium 365, et nous avons mis en place un programme de formation par projet de 10 semaines, intégrant 12 personnes à l'équipe.}}
    }


\roottitle{Éducation}
\headedsection
    {MSc Ingénierie | Double Diplôme}
    {\textsc{}}
    {
        \headedsubsection
        {\href{https://www.ensta-paris.fr/}{ENSTA Paris}, \href{https://www.ip-paris.fr/}{IP Paris}}
        {Palaiseau, France \qquad 07/2022 -- Present}
        {\bodytext{Informatique -- Robotique}}
    }

\headedsection
    {BSc Ingénierie | Double Diplôme}
    {\textsc{}}
    {
        \headedsubsection
        {\href{https://www.fem.unicamp.br/index.php/pt-br/}{FEM}, \href{https://www.unicamp.br/unicamp/universidade}{UNICAMP}}
        {Campinas, Brazil \qquad 03/2019 -- Present}
        {\bodytext{Contrôle et Automatisation -- Mécatronique}}
    }


\vspace{6mm}
\begin{multicols}{2}
    \roottitle{Langues}
    \vspace{2mm}
    \textbf{Portugais}: Langue Materne\\
    \textbf{Français}: Avancé C1\\
    \textbf{Anglais}: Avancé C1\\
    \columnbreak
    \roottitle{Skills}
    \vspace{2mm}
    \textbf{Programmation}: C, LaTeX, MATLAB, Python and VBA\\
    \textbf{Office}: Excel, PowerPoint and Word\\
    \textbf{Outres}: Altium Designer, Adobe Illustrator and Photoshop\\
\end{multicols}
\end{document}